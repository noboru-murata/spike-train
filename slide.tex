% Created 2023-06-19 Mon 12:39
% Intended LaTeX compiler: pdflatex
\documentclass[fleqn,aspectratio=1610]{beamer}
\author{Noboru Murata}
\date{\today}
\title{Estimation of Neural Connections from Multiple Spike Trains}
\subtitle{graph structure inference with nuisance inputs}
\usepackage[toc=none]{mytalk}
\addbibresource{papers.bib}
\graphicspath{{figs/},{figs/pnas/},{refs/}}
\DeclareGraphicsExtensions{.pdf,.png,.eps,.jpg}
\institute{\url{https://noboru-murata.github.io/}}
\hypersetup{
 pdfauthor={Noboru Murata},
 pdftitle={Estimation of Neural Connections from Multiple Spike Trains},
 pdfkeywords={multiple spike trains, stochastic modeling, graph inference},
 pdfsubject={based on N. Murata \& Amari 1999, https://doi.org/10.1016/S0165-1684(98)00206-0},
 pdfcreator={Emacs 28.2 (Org mode 9.6.4)}, 
 pdflang={English}}
\begin{document}

\maketitle
\begin{frame}{Outline}
\tableofcontents
\end{frame}


\section{Introduction}
\label{sec:orgad5538c}
\subsection{motivated problem}
\label{sec:org84cdec3}
\begin{frame}[label={sec:orgad5d0a9}]{Background}
\begin{itemize}
\item estimating neural connections

\begin{itemize}
\item understand functions of biological systems
\item investigate learning/adaptation mechanisms
\end{itemize}
\end{itemize}

\smallskip
\begin{itemize}
\item typical methods for measuring brain activities

\begin{itemize}
\item fMRI
(functional magnetic resonance imaging)
\item MEG
(magnetoencephalography)
\item EEG
(electroencephalography)
\item TPE
(two-photon excitation microscopy)
\item \alert{multi-electrode recording}
\end{itemize}
\end{itemize}

\smallskip
\begin{itemize}
\item different resolutions in
\begin{itemize}
\item time (oxygen consumption - neuron firing)
\item space (brain mapping - synaptic connections)
\end{itemize}
\end{itemize}
\end{frame}

\begin{frame}[label={sec:orga9ea4aa}]{Multi-Electrode Recording}
\begin{columns}
\begin{column}{0.5\columnwidth}
\includegraphics[width=\linewidth]{electrode}
\begin{quote}
\tiny
by courtesy of Dr. Tatsuno
at University of Lethbridge
\end{quote}
\end{column}
\begin{column}{0.5\columnwidth}
activities of individual neurons

\begin{itemize}
\item multiple neurons\\[0pt]
(tens - hundreds)
\item long term measurement\\[0pt]
(several hours - several days)
\end{itemize}
\end{column}
\end{columns}
\end{frame}

\begin{frame}[label={sec:org9341ed7}]{Spike Trains}
\begin{columns}
\begin{column}{0.6\columnwidth}
\centering
multi-variate point process\\[5pt]
\rotatebox{90}{\hspace*{.2\linewidth}neurons}
\includegraphics[width=.9\linewidth]{spiketrain}\\
times
\end{column}
\begin{column}{0.4\columnwidth}
rearranged as binary sequence
indicating states of neurons
\begin{itemize}
\item 0: resting
\item 1: firing
\end{itemize}
\bigskip

multi-variate binary time series contains
information of neural interactions
\end{column}
\end{columns}
\end{frame}

\begin{frame}[label={sec:orge6d5ce6}]{Graph Structure Inference}
\begin{columns}
\begin{column}{0.5\columnwidth}
\begin{center}
\includegraphics[width=.4\linewidth]{synapse}
\(\displaystyle\leftrightarrows\atop\strut\)
\includegraphics[width=.25\linewidth]{edge}\\[0pt]
\includegraphics[width=.8\linewidth]{graph}
\end{center}
\end{column}
\begin{column}{0.5\columnwidth}
mathematical representation -- \\[0pt]
directed graph
\begin{itemize}
\item node: neuron
\item edge: synaptic connection
\end{itemize}
\bigskip
\begin{alertblock}{Objective}
estimate weights of edges
from binary time series at nodes
\end{alertblock}
\end{column}
\end{columns}
\end{frame}

\begin{frame}[label={sec:org7b65202}]{Prior Works}
typical methods for analysis
\begin{itemize}
\item pair-wise:

\begin{itemize}
\item cross-correlation

(e.g. \cite{WilsonMcNaughton1994})
\item joint peri-stimulus time histogram
(e.g. \cite{ItoTsuji2000})
\end{itemize}
\item graph-based:

\begin{itemize}
\item sparse inverse covariance matrix
(e.g. \cite{FriedmanHastieTibshirani2008})
\item digraph Laplacian
(e.g. \cite{Noda_etal2014})
\end{itemize}
\item higher-order:

\begin{itemize}
\item information geometric measure
(e.g. \cites{NakaharaAmari2002,TatsunoFellousAmari2009})
\item Granger causality
(e.g. \cite{Kim_etal2011})
\end{itemize}
\end{itemize}
\end{frame}

\subsection{issues to be solved}
\label{sec:orga1c5311}
\begin{frame}[label={sec:orgd66ecb3}]{Issues to Be Solved}
\begin{itemize}
\item pseudo correlation caused by
\begin{itemize}
\item higher-order effects
\item dynamical systems
\end{itemize}
\item influence from unobserved neurons
\item directed excitatory/inhibitory connections
\end{itemize}
\end{frame}

\begin{frame}[label={sec:orga37f925}]{Pseudo Correlation}
correlation coefficient:
statistics for analyzing relation of two random variables
\medskip
\begin{columns}
\begin{column}[t]{0.5\columnwidth}
\centering
connections \\[4pt]
\includegraphics[width=.6\linewidth]{connection}

\begin{itemize}
\item no direct relation exists
\item two nodes are connected with the same node
\end{itemize}
\pause
\end{column}
\begin{column}[t]{0.5\columnwidth}
\centering
pseudo-correlation \\[4pt]
\includegraphics[width=.6\linewidth]{pseudoconnection}

\begin{itemize}
\item spurious relation appears
\end{itemize}
\end{column}
\end{columns}
\begin{alertblock}<3>{Pseudo correlation}
a common problem in complex network analysis
\end{alertblock}
\end{frame}

\begin{frame}[label={sec:org3e1977e}]{Delayed Pseudo-Correlation}
delayed correlation coefficient:
statistics for analyzing time series

\medskip
\begin{columns}
\begin{column}[t]{0.5\columnwidth}
\centering
\includegraphics[width=.7\linewidth]{longterm}\\[0pt]
\includegraphics[width=.5\linewidth]{longtermeffect}
\begin{itemize}
\item appropriate intervals have to be considered
\item information propagates multiple paths
\item spurious relation appears
\end{itemize}
\end{column}
\begin{column}[t]{0.5\columnwidth}
\centering
\visible<2->{\includegraphics[width=.7\linewidth]{shortterm}\\
\includegraphics[width=.25\linewidth]{shorttermeffect}}
\begin{itemize}
\item <2-> consider short intervals?
\end{itemize}
\visible<3>{\includegraphics[width=.6\linewidth]{isi}}
\begin{itemize}
\item <3-> spike intervals are random
\end{itemize}
\end{column}
\end{columns}
\end{frame}

\begin{frame}[label={sec:orgb935d7f}]{Our Contribution}
a mathematical framework for treating

\begin{itemize}
\item pseudo correlation caused by 
higher-order effects and dynamical systems
\item influence from unobserved neurons
\item directed excitatory/inhibitory connections
\end{itemize}

\pause
\bigskip
\begin{alertblock}{Main contribution}
solve those problems with simple mathematical tricks
\end{alertblock}
\end{frame}


\section{Problem Formulation}
\label{sec:org9de027e}
\subsection{mathematical model}
\label{sec:orge11b01d}
\begin{frame}[label={sec:orge41df7f}]{Notations}
\begin{itemize}
\item indeces
\begin{itemize}
\item \(i\in\{1,2,\dotsc,N\}\):
index of neurons
\item \(t\in\{1,2,\dotsc,T\}\):
discrete time of measurement
\item \(t_{\Delta}=[t-\Delta,\dotsc,t-1]\):

interval for delayed correlation
\end{itemize}
\item states
\begin{itemize}
\item \(X_{i}(t)\in\{0,1\}\):
state of neuron \(i\) at time \(t\)
\item \(X_{i}[t_{\Delta}]\in\{0,1\}\):
state of neuron \(i\) in interval \(t_{\Delta}\)
\item \(U_{i}(t)\in\mathbb{R}\):
internal state of neuron \(i\) at time \(t\)
\end{itemize}
\item connections
\begin{itemize}
\item \(w_{ij}\in\mathbb{R}\):
synaptic connection from neuron \(j\) to neuron \(i\)
\item \(\lambda_{ij}\in\mathbb{R}\):
pseudo connection from neuron \(j\) to neuron \(i\)
\end{itemize}
\end{itemize}
\end{frame}
\begin{frame}[label={sec:org3836cf9}]{Internal State}
weighted sum of inputs from unobserved/observed neurons
\begin{columns}
\begin{column}{0.7\columnwidth}
\begin{align}
  &U_i(t)
    = B_i(t) + \sum_{j=1}^{N} \lambda_{ij} X_j[t_\Delta],\\
  &\quad B_{i}(t): \text{nuisance inputs from unobserved neurons}\\
  &\quad \lambda_{ij}: \text{\alert{pseudo connection} including
    undirect paths}
\end{align}
\end{column}
\begin{column}{0.3\columnwidth}
\begin{center}
\includegraphics[width=\linewidth]{longterm}
\\[10pt]
\includegraphics[width=\linewidth]{longtermeffect}
\end{center}
\end{column}
\end{columns}
\begin{alertblock}{Remarks}
\begin{itemize}
\item signal from neuron \(j\) has several paths
\item \(\lambda_{ij}\) includes direct and undirect connections
\end{itemize}
\end{alertblock}
\end{frame}

\begin{frame}[label={sec:org2d3272f}]{Neuron Firing}
stochastic dependency on internal state:
\begin{columns}
\begin{column}{0.65\columnwidth}
\begin{align}
  &\Pr\bigl(X_i(t)=1\bigr)
    = \Phi_{\sigma^{2}}\bigl(U_i(t)\bigr),\\
  % &\qquad
  %   \phi_{\sigma^2}(z)
  %   =\frac{1}{\sqrt{2\pi\sigma^2}}\exp{\Bigl(-\frac{z^2}{2\sigma^2}\Bigr)},\\
  &\qquad
    \Phi_{\sigma^{2}}: \text{cdf of }\mathcal{N}(0,\sigma^{2}).
\end{align}
\end{column}
\begin{column}{0.35\columnwidth}
\centering
\footnotesize
\rotatebox{90}{\hspace*{15pt}firing probability}
\includegraphics[width=.9\linewidth]{probit}\\[0pt]
internal state
\end{column}
\end{columns}

\begin{alertblock}{Assumption}
\begin{itemize}
\item we assume a probit model, where \(\Phi_{\sigma^{2}}\) is the integral of
\begin{equation}
  \phi_{\sigma^2}(z)
  =\frac{1}{\sqrt{2\pi\sigma^2}}\exp{\Bigl(-\frac{z^2}{2\sigma^2}\Bigr)}
\end{equation}
\end{itemize}
\end{alertblock}
\end{frame}

\begin{frame}[label={sec:orgb1a8274}]{Model Description}
\begin{itemize}
\item internal state
\begin{align}
  &U_i(t)
    = B_i(t) + \sum_{j=1}^{N} \lambda_{ij} X_j[t_\Delta],\\
  &\qquad B_{i}(t): \text{nuisance inputs},\\
  &\qquad \lambda_{ij}: \text{pseudo connection}.
\end{align}
\item neuron firing
\begin{align}
  &\Pr\bigl(X_i(t)=1\bigr)
    = \Phi_{\sigma^{2}}\bigl(U_i(t)\bigr),\\
  &\qquad
    \phi_{\sigma^2}(z)
    =\frac{1}{\sqrt{2\pi\sigma^2}}\exp{\Bigl(-\frac{z^2}{2\sigma^2}\Bigr)},\\
  &\qquad
    \Phi_{\sigma^{2}}:
    \text{cdf of \(\mathcal{N}(0,\sigma^{2})\),
    integral of \(\phi_{\sigma^2}\).}
\end{align}
\end{itemize}
\end{frame}

\subsection{estimation method}
\label{sec:org3c78430}
\begin{frame}[label={sec:orga8a99e4}]{Removal of Nuisance Effect}
\begin{block}{First step}
\begin{itemize}
\item remove nuisance input \(B\) and estimate pseudo connection \(\lambda\)
\begin{align}
  U_i(t)
  &= \alert{B_i(t)} +
    \sum_{j=1}^{N} \alert{\lambda_{ij}} X_j[t_\Delta].
\end{align}
\end{itemize}
\end{block}
\begin{center}
\includegraphics[width=.5\linewidth]{partialgraph}
\end{center}
\end{frame}

\begin{frame}[label={sec:orgf03922d}]{Property of Sum of Random Variables}
\begin{theorem}[]\label{sec:orgce2353b}
Let \(X\) and \(Y\) be independent random variables.
For any function \(g\), we have
\begin{align}
  \mathbb{E}[g(X+Y)]
  &= \mathbb{E}\bigl[h\bigl(X+\mathbb{E}[Y]\bigr)\bigr],\\
  \intertext{where \(f_{Y}\) is the pdf of \(Y\) and}
  f^{-}_{Y}(x) &= f_Y(\mathbb{E}[Y]-x),\\
  h &= g*f^{-}_{Y}.
\end{align}
\end{theorem}
A special case is discussed in \cite{Hyvaerinen2002}.
\end{frame}

\begin{frame}[label={sec:org99dc8fc}]{Special Case of Gaussian}
\begin{corollary}[]\label{sec:orge561c7e}
If function \(g\) is \(\Phi_{\sigma^{2}}\)
and random variable \(X\) is constant value \(x\),
and probability density function \(f_Y\) is Gaussian
with mean \(\mathbb{E}[Y]\) and variance \(\tau^{2}\), we have
\begin{align}
  \mathbb{E}[\Phi_{\sigma^{2}}(x+Y)]
  &=\Phi_{\sigma^{2}+\tau^{2}}\bigl(x+\mathbb{E}[Y]\bigr).
\end{align}
\end{corollary}
\begin{center}
\(\Phi_{\sigma^{2}}\)\hspace{.29\linewidth}
\(\phi_{\tau^{2}}\)\hspace{.24\linewidth}
\(\Phi_{\sigma^{2}\!+\!\tau^{2}}\)\\[0pt]
\includegraphics[width=.9\linewidth]{convolution}    
\end{center}
\end{frame}

\begin{frame}[label={sec:orgbdac24f}]{Conditinal Expectation of State}
\begin{itemize}[<+->]
\item consider the case of \(X_j[t_{\Delta}]\!=\!1\),
\begin{align}
  U_{i}(t\mid X_j[t_{\Delta}]\!=\!1)
  &= B_i(t) + \lambda_{ij}X_j[t_{\Delta}] +
    \sum_{k \neq j}\lambda_{ik}X_k[t_{\Delta}]\\
  &= \lambda_{ij} + C_{ij}(t\mid X_j[t_{\Delta}]\!=\!1\bigr).
\end{align}
\item let us apply the corollary for calculating conditional expectation
\begin{align}
  \mathbb{E}\bigl[X_i(t) \mid X_j[t_{\Delta}]\!=\!1\bigr]
  &= \mathbb{E}\bigl[\Phi_{\sigma^2}\bigl(U_{i}(t\mid X_j[t_{\Delta}]\!=\!1)\bigr)\bigr]\\
  &= \mathbb{E}\bigl[\Phi_{\sigma^2}\bigl(\lambda_{ij} +
    C_{ij}(t\mid X_j[t_{\Delta}]\!=\!1)\bigr)\bigr]\\
    % \mathbb{E}\bigl[\Phi_{\sigma^2}\bigl(\lambda_{ij} + C_{ij}(t)\bigr)\bigr]
  &= \Phi_{\rho^2}(\lambda_{ij} + \bar{C}_{ij}),
\end{align}
where we assume \(C_{ij}\sim\mathcal{N}(\bar{C}_{ij},\tau^{2})\)
and \(\rho^2 = \sigma^2+\tau^2\).
\end{itemize}
\end{frame}

\begin{frame}[label={sec:org14ba647}]{Conditional Expectation of Internal State}
\begin{itemize}[<+->]
\item for binary random variables, the following holds
\begin{align}
  \mathbb{E}\bigl[X_i(t) \mid X_j[t_{\Delta}]\!=\!1\bigr]
  = \Pr(X_i(t)\!=\!1 \mid X_j[t_{\Delta}]\!=\!1).
\end{align}
\item therefore, we obtain
\begin{align}
  \Phi_{\rho^2}(\lambda_{ij} + \bar{C}_{ij})
  &= \Pr(X_i(t)=1 \mid X_j[t_{\Delta}]\!=\!1),\\
  \Leftrightarrow\quad
  \lambda_{ij} + \bar{C}_{ij}
  &= \rho\cdot\Phi^{-1}_{1}\bigl(\Pr(X_i(t)\!=\!1 \mid X_j[t_{\Delta}]\!=\!1)\bigr).
\end{align}
\end{itemize}
\end{frame}

\begin{frame}[label={sec:org43915d6}]{Difference of Conditional Expectation}
\begin{itemize}
\item consider the both cases of \(X_j[t_{\Delta}]=1\) and \(X_j[t_{\Delta}]=0\),
\begin{align}
  U_{i}(t\mid X_j[t_{\Delta}]\!=\!1)
  &= \lambda_{ij} +
    C_{ij}(t\mid X_j[t_{\Delta}]\!=\!1),\\
  U_{i}(t\mid X_j[t_{\Delta}]\!=\!0)
  &= \phantom{\lambda_{ij} + {}}
    C_{ij}(t\mid X_j[t_{\Delta}]\!=\!0).
\end{align}
\end{itemize}
\pause
\begin{alertblock}{Assumption}
\begin{equation}
  C_{ij}(t\mid X_j[t_{\Delta}]\!=\!1),
  C_{ij}(t\mid X_j[t_{\Delta}]\!=\!0)
  \sim\mathcal{N}(\bar{C}_{ij},\tau^{2})
\end{equation}
\end{alertblock}
\pause
\begin{itemize}
\item then we obtain
\begin{align}
  \lambda_{ij} + \bar{C}_{ij}
  &= \rho\cdot\Phi^{-1}_{1}\bigl(\Pr(X_i(t)\!=\!1 \mid X_j[t_{\Delta}]\!=\!1)\bigr),\\
  %% \phantom{\lambda_{ij} +}
  \bar{C}_{ij}
  &= \rho\cdot\Phi^{-1}_{1}\bigl(\Pr(X_i(t)\!=\!1 \mid X_j[t_{\Delta}]\!=\!0)\bigr).
\end{align}
\end{itemize}
\end{frame}

\begin{frame}[label={sec:orgee6a9f3}]{Estimation of Pseudo Connection}
\begin{itemize}
\item estimator of pseudo connection
\begin{align}
  \lambda_{ij}
  &=\rho\bigl\{\Phi^{-1}_{1}\bigl(\Pr(X_i(t)\!=\!1 \mid X_j[t_{\Delta}]\!=\!1)\bigr)\\
  &\qquad\qquad
    - \Phi^{-1}_{1}\bigl(\Pr(X_i(t)\!=\!1 \mid X_j[t_{\Delta}]\!=\!0)\bigr)
    \bigr\}.
\end{align}
\item empirical estimates of conditional probability
\begin{align}
  \Pr(X_i(t)\!=\!1 \mid X_j[t_{\Delta}]\!=\!1)
  &= \frac{1}{Z}\sum_{t} X_i(t \mid X_j[t_\Delta]\!=\!1),\\
  \Pr(X_i(t)\!=\!1 \mid X_j[t_{\Delta}]\!=\!0)
  &= \frac{1}{Z'}\sum_{t} X_i(t \mid X_j[t_\Delta]\!=\!0).
\end{align}
\end{itemize}
\end{frame}

\begin{frame}[label={sec:orgdde1ee4}]{Decomposition of Pseudo Connection}
\begin{block}{Second step}
\begin{itemize}
\item decompose pseudo connections \(\lambda\)
with direct connections \(w\)
\end{itemize}
\centering
\includegraphics[width=.9\linewidth]{decomposition}
\end{block}
\pause
\begin{itemize}
\item consider an expansion with appropriate \(\delta,\delta'\) (delay time)
\begin{align}
  \lambda_{ij}
  &=w_{ij}\\
  % &+\sum_{k}w_{ik}\alert{\Pr(X_k(t)\!=\!1 \mid X_j[t_{\delta}]\!=\!1)}\\
&+\sum_{k}\!w_{ik}\alert{\Pr(X_k(t\!-\!\delta)\!=\!1 \mid X_j(t\!-\!\delta')\!=\!1)}\\
% &+\sum_{k,l}\!w_{ik}\alert{\Pr(X_k(t)=1 \mid X_l(t-{\delta})=1)}\\
% &\phantom{+\sum_{k,l}\!w_{ik}}\times
%   \alert{\Pr(X_l(t-\delta)=1 \mid X_j(t-2\delta)=1)}\\
&+\text{(higher order terms)}.
\end{align}
\end{itemize}
\end{frame}

\begin{frame}[label={sec:orgbab3520}]{Decomposition}
\begin{itemize}
\item introduce a virtual probability with an appropriate interval \(t_{\delta}\)
\begin{equation}
  \theta_{ij}
  = \Pr(X_i(t)\!=\!1 \mid X_j[t_{\delta}]\!=\!1).
\end{equation}
\item we obtain an expansion of \(\lambda\) as
\begin{align}
  \lambda_{ij}
  &=w_{ij}+
    \sum_k\!w_{ik}\theta_{kj}+
    \sum_{k,l}\!w_{ik}\theta_{kl}\theta_{lj}+
    \sum_{k,l,m}\!w_{ik}\theta_{kl}\theta_{lm}\theta_{mj}+
    \dotsb.
\end{align}
\item this expression gives a simple matrix form
\begin{align}
  \Lambda
  &= W(I+\Theta+\Theta^{2}+\Theta^{3}+\cdots)
  &&\text{\triangleright\,Neumann series}\\
  &= W(I-\Theta)^{-1},
\end{align}
where \(W=(w_{ij})\) and \(\Theta=(\theta_{ij})\).
\end{itemize}
\end{frame}

\begin{frame}[label={sec:orgfa544c5}]{Estimation of Virtual Probability}
\begin{itemize}
\item relation between \(\theta\) and \(w\):
\begin{align}
  \theta_{ij}
  &= \Pr(X_i(t)\!=\!1 \mid X_j[t_{\delta}]\!=\!1)
  &&\text{\triangleright\,use expectation form}\\
  &= \mathbb{E}\bigl[\Phi_{\sigma^{2}}(w_{ij}+C_{ij}')\bigr]
  &&\text{\triangleright\,\(t_{\delta}\) is small enough}\\
  % to exclude undirect effect.
  &= \Phi_{\rho^2}\bigl(w_{ij}+\mathbb{E}[C'_{ij}]\bigr)
  &&\text{\triangleright\,by the corollary}
\end{align}
\end{itemize}
\pause
\begin{alertblock}{Assumption}
\begin{equation}
  C_{ij}'\sim\mathcal{N}(\bar{C}_{ij},\tau^{2})
\end{equation}
\end{alertblock}
\pause

\begin{itemize}
\item calculate \(\theta\) by using \(w\) as
\begin{align}
  \theta_{ij}
  &=\Phi_{\rho^2}(w_{ij}+\bar{C}_{ij}),\\
  % \mathbb{E}[C'_{ij}]
  \bar{C}_{ij}
  &= \rho\cdot\Phi^{-1}_{1}\bigl(\Pr(X_i(t)\!=\!1 \mid X_j[t_{\Delta}]\!=\!0)\bigr).
\end{align}
\end{itemize}
\end{frame}

\begin{frame}[label={sec:org5627d79}]{Estimation of Neuron Type}
\begin{block}{Third step}
\begin{itemize}
\item estimate types of neurons consistent with data:
\begin{itemize}
\item excitatory neurons - \alert{positive connections only}
\item inhibitory neurons - \alert{negative connections only}
\end{itemize}
\end{itemize}
\end{block}
\pause
\medskip
\begin{columns}
\begin{column}{0.5\columnwidth}
\centering
\visible<3->{\includegraphics[width=.8\linewidth]{emalgorithm}\\}
\end{column}
\begin{column}{0.5\columnwidth}
treated as hidden variables \(\boldsymbol{z}\in\{0,1\}^{N}\)
\begin{equation}
  \Pr(\text{Data}\mid W,\boldsymbol{z})
  \Leftrightarrow
  \Pr(\boldsymbol{z}\mid\text{Data},W)
\end{equation}

\pause
\smallskip
use em algorithm
\parencite{Amari1995}\\[0pt]
with approximations:
\begin{itemize}
\item factorial model in data manifold
\item Gibbs sampling
\end{itemize}
\end{column}
\end{columns}
\end{frame}

\begin{frame}[label={sec:orga6d4d19}]{Proposed Algorithm}
\small
\begin{algorithmic}[1]
  \State{Input:\;\(\Lambda, \bar{C}, \boldsymbol{z}\)}
  \Function{estimateW}{\(\Lambda, \bar{C}, \boldsymbol{z}\)}
  \State{Initialization:\;
    \(\Theta^{(1)}\gets[0,1]^{N\times N}\),
    \(\Lambda^{(1)}\gets\Lambda\)}
  \For{\(\tau\gets1,T\)}
  \State{\(W^{(\tau+1)}\gets
    {\Lambda}^{(\tau)}(I-\Theta^{(\tau)})\)}
  \For{\(i\gets1,N\)}
  \For{\(j\gets1,N\)}
  \State{
    \([\hat{W}(\boldsymbol{z})^{(\tau+1)}]_{ij}
    \gets
    \begin{cases}
      z_{j}[W^{(\tau+1)}]_{ij},&[W^{(\tau+1)}]_{ij} > 0\\
      (1-z_{j})[W^{(\tau+1)}]_{ij},&[W^{(\tau+1)}]_{ij} < 0
    \end{cases}
    \)
  }
  \EndFor
  \EndFor
  \State{\(\bigl[\Theta^{(\tau+1)}\bigr]_{ij}\gets
    \Phi_1\Bigl([\hat{W}(\boldsymbol{z})^{(\tau+1)}]_{ij}+\bar{C}_{ij})\Bigr)\)}
  \State{\(\mathrm{diag}(\boldsymbol{\Theta}^{(\tau+1)})\gets 0\)}
  \Comment{update diagonal elements}
  \State{\(\Lambda^{(\tau+1)}\gets\Lambda^{(\tau)}\)}
  \State{\(\mathrm{diag}\bigl(\Lambda^{(\tau+1)}\bigr)\gets
    \mathrm{diag}\bigl(\Lambda^{(\tau)}\Theta^{(\tau+1)}\bigr)\)}
  \Comment{update diagonal elements}
  \EndFor
  \EndFunction
  \State{Output:\;\(\hat{W}(\boldsymbol{z})\)}
\end{algorithmic}
\end{frame}


\section{Numerical Examples}
\label{sec:org9a3db70}
\subsection{synthetic data analysis}
\label{sec:orgcb548e3}
\begin{frame}[label={sec:org750a8a4}]{Synthetic Data Analysis}
\begin{columns}
\begin{column}{0.5\columnwidth}
\centering
\includegraphics[width=.68\linewidth]{experiment}
\end{column}
\begin{column}{0.5\columnwidth}
Izhikevich's neuron model\\[0pt]
\parencite{Izhikevich2003}
\begin{itemize}
\item \(N=33\) out of \(100\) neurons
\item excitatory:inhibitory = 80\%:20\%
\item \(w_{ij}\sim\mathrm{Unif}[-10,10]\)
\item \(\#\{w_{\cdot i}\}\leq 10\)
\end{itemize}
\centering
\includegraphics[width=.8\linewidth]{spiketrain}
\end{column}
\end{columns}
\end{frame}
\begin{frame}[label={sec:org77fd14f}]{Estimation Result}
\begin{columns}
\begin{column}{0.5\columnwidth}
\centering
true\\[0pt]
\includegraphics[width=.75\linewidth]{graph_cor}
\end{column}
\begin{column}{0.5\columnwidth}
\centering
estimated\\[0pt]
\includegraphics[width=.75\linewidth]{graph_est}
\end{column}
\end{columns}
\begin{alertblock}{remarks}
\begin{itemize}
\item estimation is scale indeterminate
\item inhibitory connections are difficult to estimate
\end{itemize}
\end{alertblock}
\end{frame}

\begin{frame}[label={sec:org49fa4c1}]{Performance}
\begin{columns}
\begin{column}{0.5\columnwidth}
\centering
sensitivity
\includegraphics[width=.75\linewidth]{fig_100net1}
\end{column}
\begin{column}{0.5\columnwidth}
\centering
Kendall coefficient
\includegraphics[width=.75\linewidth]{fig_100net2}
\end{column}
\end{columns}
\begin{alertblock}{remarks}
\begin{itemize}
\item estimation accuracy gets better if neuron types are given
\item order of weights is estimated with sufficient accuracy
\end{itemize}
\end{alertblock}
\end{frame}
\begin{frame}[label={sec:org1fac40a}]{Sensitivity vs Interval Size}
\begin{columns}
\begin{column}{0.5\columnwidth}
\centering
sensitivity
\includegraphics[width=.75\linewidth]{fig_100net}
\end{column}
\begin{column}{0.5\columnwidth}
\centering
spike interval
\includegraphics[width=.95\linewidth]{isi}
\end{column}
\end{columns}
\begin{alertblock}{remark}
\begin{itemize}
\item sensitivity is affected by choice of
correlation interval
\end{itemize}
\end{alertblock}
\end{frame}
\subsection{real data analysis}
\label{sec:org8064bf4}
\begin{frame}[label={sec:org74d7fbe}]{Real Data Analysis}
memory trace replay
\parencites{WilsonMcNaughton1994,TatsunoLipaMcNaughton2006}

\begin{itemize}
\item purpose: 
examine the hyposesis ``the replay of activity patterns during sleep 
plays an important role in the consolidation process of memory''
\item measurements:
\begin{itemize}
\item pre-task: activity of control
\item task: activity in learning stage
\item post-task: activity in non-REM stage
\end{itemize}
\end{itemize}
\end{frame}

\begin{frame}[label={sec:org42415ca}]{Estimation Result}
\begin{columns}
\begin{column}{0.33\columnwidth}
\centering
pre-task \\[10pt]
\includegraphics[width=.9\linewidth,trim=94 87 66 73,clip]{rslt_Rat8000_Seq_pre}
\end{column}
\begin{column}{0.33\columnwidth}
\centering
task \\[10pt]
\includegraphics[width=.9\linewidth,trim=94 87 66 73,clip]{rslt_Rat8000_Seq_task}
\end{column}
\begin{column}{0.33\columnwidth}
\centering
post-task \\[10pt]
\includegraphics[width=.9\linewidth,trim=94 87 66 73,clip]{rslt_Rat8000_Seq_post}
\end{column}
\end{columns}
\begin{alertblock}{remarks}
\begin{itemize}
\item some connections changed at task period are retained at
post-task period (e.g. 8,11,12,20)
\item result should be discussed from the viewpoint of biology
\end{itemize}
\end{alertblock}
\end{frame}

\section{Conclusion}
\label{sec:orga1bd4d7}
\begin{frame}[label={sec:org12307c4}]{Concluding Remarks}
we consider an approach to solve the following problems
\begin{itemize}
\item pseudo correlation caused by higher-order effect
\item influence from unobserved neurons
\item directional excitatory/inhibitory connections
\end{itemize}

possible extension would be
\begin{itemize}
\item estimating the number of connections
\item estimating activation functions of individual neurons
\item applying other real-world data
\end{itemize}
\end{frame}

\begin{frame}[allowframebreaks]{References}
\printbibliography[heading=none]
\end{frame}
\end{document}